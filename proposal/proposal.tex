\documentclass[11pt]{article}
\bibliographystyle{siam}

\title{The neural basis of loss aversion in decision making under risk}
\author{
  Kong, Victor\\
  \texttt{VictorKong94}
  \and
  Li, Ce\\
  \texttt{karenceli}
  \and
  Liu, Anna\\
  \texttt{liuanna}
  \and
  Qin, Weidong\\
  \texttt{j170382276}
  \and
  Xia, Yunfei\\
  \texttt{yfxia}
}

\begin{document}
\maketitle

By exploring the published fMRI paper \textit{The neural basis of loss aversion in decision making under risk} and the accompanying data ($http://openfmri.s3.amazonaws.com/tarballs/ds005_raw.tgz$), our team decides to investigate the neural response when individuals decided whether to accept or reject gambles that offered a 50/50 chance to win or lose. By the prospect theory by Daniel Kahneman, the paper examines the loss aversion by analyzing how individual's brains react differently when they are facing potential gains or loss. Humans are more sensitive to loss than gains. The goal of the study is whether loss aversion reflects the engagement of distinct emotional processes when potential losses are considered. The paper investigates whether individual differences in brain activity during decision-making are related to individual differences in behavior. Researchers used fMRI images to analyze the whole brain and to identify regions where the neural response to gains or losses was correlated with loss aversion. The paper illustrates how neuroimaging is used to directly test the prediction from prospect theory that risk aversion for mixed gambles can be attributed to enhanced sensitivity to losses. 

\bibliography{proposal}

\end{document}
