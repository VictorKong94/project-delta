% 3 Logistic Regression

\par This analysis uses the behavioral data, particularly the regressors
parametric gain, parametric loss, and the distance from indistance. Two logistic
regression models are fit for each run: one that uses the distance from
indifference, and one that does not. The reasoning behind this decision is that
the distance from indifference is partially dependent on both the parametric
gain and the parametric loss, so including it may underestimate the effect of
its precursors.

\par \indent For the purposes of logistic regression, the values of these
variables are placed into a design matrix. The design matrix for each run
contains as many rows as there are trials associated with that run, and one more
column than the number of regressors used. In both the case that the distance
from indifference is included and the case that the distance from indifference
is excluded, the first column contains all ones, the second contains the values
of potential gain, and the third the values of the potential loss. In the case
that the distance from indifference is used, the values of said distance are
contained in the fourth column. Responses are taken from the sixth column of the
behavioral data matrix (which contains the subject's binary responses: simple
acceptance or declination of the wager).

\par \indent Once each logistic model was fit, the values of the behavioral loss
aversion for each voxel in the brain is computed as the additive inverse of the
regression coefficient for parametric loss divided by the regression coefficient
for the parametric gain. These values are denoted \texttt{Lambda} ($\lambda$)
and are placed aside for subsequent conjunction analysis.

\par \indent Lastly, each logistic regression model fitted has its accuracy 
accessed via computation of its misclassification rate and its statistical
significance accessed via a Wald test.
