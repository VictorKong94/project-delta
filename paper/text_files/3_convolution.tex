% 3 Convolution

\par The study is built upon an event-related experimental design, in which the
onsets are unevenly-spaced. Convolved hemodynamic response function (HRF)
predictions for parametric gain, parametric loss, and the distance from
indifference were computed using the data obtained from the condition files
\texttt{cond002.txt}, \texttt{cond003.txt}, and \texttt{cond004.txt},
respectively.

\par \indent The analysis assumes the canonical hemodynamic response function,
which has the distribution of the difference between two gamma probability
density functions: the peak (that with shape parameter 6 and scale parameter 1)
minus 0.35 times the undershoot (that with shape parameter 12 and scale
parameter 1). That is,
\[
\mathrm{canonical \ HRF} \sim \Gamma \left( 6 , 1 \right) - 0.35 \cdot \Gamma \left(
12 , 1 \right).
\]

\par \indent For each of the three conditions per run of each subject, the
neural time course is first initiated as a one-dimensional vector of zeros with
length 480, with each element representing exactly one second of the run. The
time of onset and duration for each trial is then extracted, and the
``amplitude'' for each trial is inserted into elements in which that particular
trial took place. The end result is a vector of length 480 that contains zeros
at indices that correspond to time between trials and the appropriate
``amplitudes'' for indices that correspond to time during trials.
