% 3 Diagnosis

\par The diagnostic analysis essentially has two parts: the first concerns
finding outliers in standard deviation over all voxels for each volume, while
the second concerns outliers in root-mean-square difference over all voxels
between consecutive volumes.

\par \indent The interquartile range (IQR) of a set of numerical data is defined
to be the difference between the 75th percentile and the 25th percentile of the
data. In the first part, the standard deviation over voxels for each volume is
computed and outliers are defined to be volumes whose standard deviation is
greater than or equal to 1.5 times the IQR \textit{above} than the 75th
percentile or greater than or equal to 1.5 times the IQR \textit{below} the 25th
percentile. The standard deviation values and the indices of their outliers are
saved to a separate file for analysis and future reference.

\par \indent The root-mean-square (RMS) difference here is defined to be the
square root of the mean (across voxels) of the squared difference between the
\textit{i}th volume and the (\textit{i}+1)th volume. In the second part, the RMS
difference was computed for each volume and outliers are once again defined to
be volumes whose RMS difference is greater than or equal to 1.5 times the IQR
\textit{above} the 75th percentile or greater than or equal to 1.5 times the IQR
\textit{below} the 25th percentile. The RMS values and the indices of both the
\textit{i}th volume and the (\textit{i}+1)th volume, in the case of an outlier,
are saved for analysis future reference.
