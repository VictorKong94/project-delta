\par \indent Prior to statistical analysis, we perform a series of
preprocessing steps to minimize the influence of data acquisition and 
physiological artifacts.The principle justification for using the Gaussian as a 
smoothing filter is due to its frequency response. Most convolution-based 
smoothing filters act as lowpass frequency filters and are designed to remove 
high spatial frequency frequency components from an image. If the spatial 
extent of a region of is larger than the spatial resolution, smoothing may 
reduce random noise in individual voxels and increase the signal-to-noise ratio 
within the region. Here, our 4-dimensional image data  were spatially smoothed 
using a 5 mm FWHM Gaussian kernel as suggested in the reference material.
\par \indent In addition, additional covariates may be included to account for
periodic noise present in the signal, such as heart-rate and respiration. 
Since more often than not, the effects of physiologicla noise are left
unmodeled, the aliased physiologicla artifacts may give rise to the
autocorrelated noise in fMRI data. We thus perform exploratory data analysis on 
residuals, i.e., difference between reponse vector (data in the bold image) and 
fitted values calculated from linear regression, to fit it with AR(1), AR(2) 
and model this seasonal component. 
