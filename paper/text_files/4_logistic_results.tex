\par \indent Logistic regression is performed as a supplementary analysis for
analyzing neural loss aversion and will be used in conjunction analysis. In 
the analysis, all trials were modeled using a single condition (i.e. overall
tak-related activation), and three additional orthogonal parametric regressors
were inlcluded representing:(a) the size of the potential gain, (b) the size
of the potential loss, and (c) the Euclidean distance of the gain/loss 
conbination from the diagonal of the gamble matrix (i.e., the distance from 
indiference assuming lambda=2 and a linear value function). It is noted in 
the paper this latter variable was included because of behavioral evidence 
suggesting greater diffculty making a decision for trials in which participants
had the weakest preference. The main purpose of the logistic regression is to
test how significant is the estimated coefficient in front of this latter
variable. Since the paper does not provide further discussion on the results
of this logistic regression, we decide to directly look at the output from the
logistic regression. 

\par \indent In design of the logistic regression, we select the feature gain, 
loss, response (0 or 1) behavioral data and calculate the euclidean distance 
from each gain/loss combination to the diagonal of the gamble matrix to 
perform the logistic regression. We construct logistic model according to 
the definition of logistic regression. We calculate the log likelihood and 
find the coefficients that result in the maximum log likelihood. In order 
to make the model more accurate, we add apenalty term to our model. Therefore, 
we have to estimate the tunning parameter for this penalty term using cross 
validation.

\par \indent After running a for loop for all 16 subjects that each has three
runs, we obtain a crude ovservation that the estimated coefficient in front of
the difference from indifference term vary braodly. We manually look at the 
results for each run and conclude the following observations: 
First, the p values for all 48 result text files showing the gain/loss term
are always presented in the proposed models with significant estimated 
coefficients. Second, the intercept terms are not always insignificant for 
different subjects. For all the gain/loss combination in run001, we find out 
that Subject 1, 2, 8, 13, 15 appear to be a loss-aversion person (the 
coefficient of difference from indifference is negative, meaning strong 
preference of rejecting the offer). Subject 6, 12, 14, 16 appear to be the 
more risk-like. 
Noticeably, the p value we are looking at are p values based on each person of
each run, thus lacking of a Bonderroni correction for doing multiple hypothesis 
testing.  



