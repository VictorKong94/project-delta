% TeX file: 1 Introduction

\par This study was inspired by ``The Neural Basis of Loss Aversion in Decision
Making Under Risk,'' a 2007 article by Sabrina M. Tom and her associates at the
University of California, Los Angeles \cite{tom2007neural}. The original study
it describes made several major findings that will be corroborated and further
investigated using methods of computational research and machine learning.
\par \indent One finding of interest to this study is that certain areas of the
human brain responded with increasing activity given higher potential gains, but
that no area responded similarly given higher potential losses. Furthermore,
many of the same areas responded with decreasing activity given higher potential
losses, but no area responded similarly given higher potential gain. This would
suggest that the same areas of the brain are responsible for processing possible
gains and losses, an idea that will be investigated in detail.
\par \indent This report will describe briefly attempts to reproduce particular
methods used in the original study, but will focus more on novel approaches not
employed by the previous researchers. The hope is that some of these methods
differ drastically from those that have already been used, and henceforth have
the potential to shed light on discoveries that were not possible before.
\par \indent Lastly, this study will place heavy emphasis on the theme of
reproducibility, which is defined to be how easily a line of analysis can be
replicated in its entirety, either by the original analyst or an independent
party. To achieve this, any code that is used to interpret the data, be the
outcome conclusive or not, will be well-written, documented, and reviewed prior
to publication.