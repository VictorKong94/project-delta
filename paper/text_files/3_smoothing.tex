% 3 Smoothing

\par Prior to further statistical analysis, it is a good idea to perform a few
preprocessing steps to minimize the influence of errors in data acquisition and
discrepancies that arise due to physiological artifacts. The use of a Gaussian
filter to ``smooth'' data has the effect of lowering higher frequencies, while
at the same time enhancing lower ones. The earlier diagnosis analysis showed
that outliers were nonfrequent and not out-of-hand so one way to remedy the few
problems left in the filtered data set is to undertake this step of Gaussian
smoothing. The result is data whose intrinsic spatial correlation is more
pronounced.

\par \indent The Gaussian kernel used here has a full-width-at-half-maximum
measurement of 5 millimeters, which corresponds to a standard deviation of
approximately 2.355 millimeters. It is now necessary to convert this value into
units of voxels. Each voxel of the data corresponds to a $2 \times 2 \times 2$
cubic region in space, so the following conversion gives the necessary standard
deviations $\sigma$ for smoothing by our specifications:
\[
\sigma_{smoothing} = \frac{ 5 \ \mathrm{ mm \ FWHM } }{ \sqrt{ 8 \ln{ 2 } } \
\mathrm{ FWHM } } \cdot \left[ \; 0.5 \; \; 0.5 \; \; 0.5 \; \right] \
\frac{ \mathrm{ voxels } }{ \mathrm{ mm } } \approx
\left[ \; 1.062 \; \; 1.062 \; \; 1.062 \; \right] \ \mathrm{ voxels } .
\]

