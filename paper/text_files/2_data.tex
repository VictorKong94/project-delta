% 2 Data

\par The original study recruited sixteen right-handed, English-speaking
individuals between the ages of 19 and 28. There is confusion about the sexes of
the individuals as the supplement to the published article reports seven males
and nine females, while the data made publicly available reports eight males and
eight females. It is known that all subjects were physically healthy and free of
neurological and psychiatric history, and that their informed consent was
acquired prior to their participation in the study.

\par \indent These subjects participated in three runs of 85-86 trials of a
single task. This task involved offering the subject a wager with an equal
chance of winning or losing known amounts of money. Potential gains were in even
denominations between \$10 and \$40, while potential losses were in integer
denominations between \$5 and \$20. The participants were prompted to either
strongly accept, weakly accept, weakly decline, or strongly decline the wager in
each trial. At the same time that investigators collected these behavioral
responses, data pertaining to neural activity was recorded via fMRI. This data
set is available for download from OpenFMRI.org under the name ``Mixed-gambles
task'' (accession number ds005), and offers both the neural activity data as
well as the behavioral data.

\par \indent The raw neural activity data for each run is saved under the
filename \texttt{bold.nii.gz}, each file containing 240 volumes obtained over
the course of eight minutes. Volumes were taken two seconds apart and each
further contains exactly 139264 voxels, being 64 points in length, 64 points in
width, and 34 points in depth. As per usual with blood-oxygen-level dependent
(BOLD) analysis, each voxel also corresponds to a signal at a given time. A
larger signal indicates increased oxygenated blood flow to the corresponding
area of the brain, which this paper will refer to as ``activation.''

\par \indent Some analyses performed in this investigation instead use a second
set of the BOLD data that has undergone preprocessing algorithms, including
motion correction, high-pass filtering with respect to time, and registration to
the Montreal Neurological Institute's standardized anatomical template. These
files are also provided by OpenFMRI.org and there exists one by the alias of
\texttt{filtered\_func\_data\_mni.nii.gz} for each run of the raw data.

\par \indent The behavior data for each run is stored separately under the alias
\texttt{behavdata.txt}. Each file contains a table for the run, in which each
trial of the task is stored as a row. Each row then contains seven elements,
referred to in the file as \texttt{onset}, \texttt{gain}, \texttt{loss},
\texttt{PTval}, \texttt{respnum}, \texttt{respcat}, and \texttt{RT}. The first
three elements respectively denote the time at which the trial was presented,
the proposed monetary reward given in the event the subject wins the trial, and
the proposed monetary loss suffered in the event the subject lost the trial. The
fourth element is a standardized value indicating the relative expected gain
from the wager. The fifth element represents the subject's response: \texttt{1}
for strong acceptance, \texttt{2} for weak acceptance, \texttt{3} for weak
rejection, \texttt{4} for strong rejection, and \texttt{0} for nonresponse. The
sixth element contains the patient response converted to a binary variable:
\texttt{0} for rejection, \texttt{1} for acceptance, and \texttt{-1} in the
outstanding case of nonresponse. The final element is the time it took the
subject to submit a response, measured in seconds and with precision to within a
millisecond.

\par \indent In both this study and the investigation that inspired it, an
additional explanatory variable called the \texttt{distance from indifference}
is also used. Each value of this variable is computed using the values of
\texttt{gain} and \texttt{loss} referenced above. With respect to these studies,
the gain/loss matrix is defined to be the two-dimensional array that contains
combinations of parametric gain and parametric loss. The first column from the
left contains all combinations with a potential to gain ten dollars. The second
column then holds all combinations with a potential to gain twelve dollars. The
third, fourteen dollars, and so on. Likewise, the first row from the top
contains every combination with the possibility of losing five dollars, the
second row six dollars, the third row eight dollars, and so on. The diagonal
referred to here is the one running from the top-left of the matrix to the
bottom-right, which includes all combinations in which the possible gain is
exactly equal to twice the potential loss. The so-called ``distance to
indifference'' therefore refers to the length of the orthogonal vector from the
diagonal to the point on the matrix representing the specific combination of
parametric gain and loss. In terms of \texttt{gain} and \texttt{loss}, this
value can be computed explicitly using the formula
\[
\mathrm{distance \ to \ indifference} = \frac{ \sqrt{ 2 } }{ 4 } \cdot
\mathrm{ gain } - \frac{ \sqrt{ 2 } }{ 2 } \cdot \mathrm{ loss }.
\]

\par \indent Lastly, the data for each run also includes a set of four plaintext
files: \texttt{cond001.txt}, \texttt{cond002.txt}, \texttt{cond003.txt}, and
\texttt{cond004.txt}. These essentially contain information concerning different
conditions whose manipulation is believed to have a significant effect on the
subject's behavioral response. Each of the four files contains data that is
organized into three columns. The first column consists of the trials' onset
times in seconds after the start of the run. The second column contains the
amount of time that was alloted to the subject to provide a response. In the
case of \texttt{cond002.txt}, \texttt{cond003.txt}, and \texttt{cond004.txt},
the third column contains values between -1 and 1 that correspond to each
particular trial's value of a parameter relative to the set of all possible
values for that parameter (which will be referred to as ``amplitude'' for the
purposes of this paper). These parameters are parametric gain, parametric loss, and the distance from indifference, respectively. In the special case of
\texttt{cond001.txt}, the parameter is the subject's assigned task, and because
this study required only a single task to be repeated over and over, the value
of this parameter was 1 for all trials.
