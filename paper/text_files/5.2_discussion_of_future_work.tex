% TeX file: 5.2 Discussion of Future Work
\par One idea that arose during the preliminary stages of the project that has
now become a work currently in progress is the implementation of support vector
machines. A support vector machine is a powerful algorithm used for prediction
in machine learning, often for classification purposes. In theory, it can allow
the user to draw flexible yet robust boundaries between data points given some
criterion. If properly implemented, models generated by this algorithm could
shed light on an entire class of inquiries, like what parts of the brain would
activate given increasing parametric gains and losses, and vice versa. The
mechanism by which support vector machines work make such topics natural to work
with, while also introducing an additional degree of resolution and confidence
in findings that pertain to these topics.
\par \indent Another scheme that has the potential to be even more difficult to
implement is decision trees. What decision trees lack in predictive power, they
make up many fold in interpretability. Models generated via a decision tree can
clearly and simultaneously show the principle variables responsible as well as
how multiple variables interact with each other in the prediction of an outcome.
This would greatly facilitate the investigation of such questions as how much
does a brain area's activity have to change before the subject changes his/her
answer, or how great of an increase in potential gain is necessary for the brain
to recognize it over a given potential loss.
