\par \indent Our main difficulty regarding logistic regression is on how to 
aggregate 48 text files together and draw conclusions on the corresponding 16 
subjects of three runs respectively. Since we are only extracting beta 
estimates for each subject and each run, it makes sense to further calculate
the predicted probablities of accepting the gain/loss combination in the end.
However, without the priori information in the behavdata.txt on whether the
subject take/reject the offer, the precision of the logistic regression 
cannot be established. 

\par \indent Another major issue we have overlooked is the problem of doing 
multiple testing, we are doing statistical inferences simultaneously for all 
16 subjects. Without correction, this might lead to incorrectly rejecting
the null hypothesis. We did not include Bonferroni correction in the scripts
when doing hypothesis testing. 

\par \indent The way we design the logistic regression might be different 
from what the paper suggested doing due to lack of information. Another 
possible interesting paramter to consider it RT, which possibly stands for
'Response Time'. The paper draws conclusion that participants are slower
to decide whether or not to accept these gambles. We can include this as
another explanatory variable in the logistic regression and observe the 
changes in the estimated parmaters and possiblly, perform model selction.